\chapter{Introduction}\label{sec:intro}
As part of the course IMT4202 / IMT4305 the project task was to create a PoC
software capable of performing a set of defined tasks using the skills learned
during the course.

I set to solve this using a mixture of Python and Matlab, where Matlab was
implemented in Python using bindings through several libraries readily available
for the public.

\section{Problem description}\label{sec:}
The task handed out consisted of several sub-problems / tasks to be met. Some
are fairly easy and others took some time.  In addition there are several
methods proposed to further the tool in terms of capability.

The tasks set forth by the task was the following:
\begin{multicols}{2}
\begin{enumerate}
	\item Create a database of min, 10 images.
	\item Maintain a set of noisy images.

	\item Apply Fourier transform on image.
	\item Reconstruct image from phase and magnitude.
	\item Reconstruct image from its Fourier transform
	\item Detect noise in frequency domain.
	
	\item Filter noise in the image.
	\item Blur image (Gaussian, ideal, Butterworth)
	\item Sharpen image (Gaussian, ideal, Butterworth)
	
	\item Display real, imaginary and magnitude of image.
	\item Compare original and filtered image for similarities.
\end{enumerate}
\end{multicols}


I've focused my efforts mainly on performing filtering tasks and less on
methods of comparison due to time restraints, and instead I've proposed some
methods for comparison as future work proposals.  In order to generate certain
filters for blurring and sharpening I've relied upon the works of the Department
of Computer Science at UC Reginas.  Their work has become a well known source
for generating Gaussian, ideal and Butterworth filter in Matlab scripts. Their
work has been converted into the Python language to extend it's audience.

The source code for this software, in its entirety is located at my publicly
accessible GitHub account\cite{git}, \url{https://github.com/Magnus1990P/imt4305/}.


\section{Overview software}
The software is based on a set of Python libraries and third-party software /
libraries.

Mainly the programming language is Python v2.7 and the version used for testing
is Python v2.7.12. The libraries utilised in the software mainly focused on
default Python libraries available from Python repository. In addition to these
libraries I've ported the default Matlab script for calculating the DFTUV of an
image and the high-pass or low-pass filter with either an "Ideal", "Gaussian" or
"Butterworth" cut-off curve. These Matlab script was created by The Department
of Computer Science at UC Regina.  In addition I've used the textbook, "Digital
Image Processing by Gonzalez and Woods\cite{fip}, for many of the aspects
implemented in the program.

The three Matlab scripts from UC Regina can be found at the following links.
\begin{itemize}
	\item DFTUV.m, generating a padded meshgrid.\cite{ucr_dftuv}
		\url{http://www.cs.uregina.ca/Links/class-info/425-nova/Lab5/M-Functions/dftuv.m}
	\item lpfilter.m Low-pass filter.\cite{ucr_lpfilter}
		\url{http://www.cs.uregina.ca/Links/class-info/425-nova/Lab5/M-Functions/lpfilter.m}
	\item hpfilter.m High-pass filter.\cite{ucr_hpfilter}
		\url{http://www.cs.uregina.ca/Links/class-info/425-nova/Lab5/M-Functions/hpfilter.m}
\end{itemize}

The standard Python libraries used in the software are.
\begin{itemize}
\begin{multicols}{3}
	\item OpenCV v2, cv2
	\item numpy
	\item sys
	\item scipy
	\item scipy.misc
	\item scipy.signal
	\item matplotlib.pyplot
\end{multicols}
\end{itemize}


