\chapter{Conclusion}\label{sec:conclusion}

The software developed as part of this project is a proof of concept and limited
in its functionality. However several of the functions employed in the software
work very well, especially the filtering of local hot spots in the image.

The tasks set forth by the task description is met, but lacks in the part of
performing comparison between the images, but that is discussed previously in
this report and in section \ref{sec:futurework}. Several methods and
improvements are discussed for enhancing the ability to compare and quantify
different aspects of the images.


\begin{table}[h]
	\begin{tabular}{p{0.5\linewidth} r p{0.4\linewidth}}
		\bf{Task} & \bf{Status} & \bf{Remark} \\[2mm]\hline
	Create a database of min, 10 images.					& $\boxed{\checkmark}$& \\
	
	Maintain a set of noisy images.								& $\boxed{\checkmark}$& 
		Four images.\\

	Apply Fourier transform on image.							& $\boxed{\checkmark}$&
		Automatically performed during load \\

	Reconstruct image from phase and magnitude.		& $\boxed{\checkmark}$&
		Done automatically on load\\
	
	Reconstruct image from its Fourier transform.	& $\boxed{\checkmark}$&
		Done via the reversal processing\\
	
	Detect noise in frequency domain.							& $\boxed{\checkmark}$& Done
		using local comparison with the mean value of image as reference value.\\
	
	Filter noise in the image.										& $\boxed{\checkmark}$& Low-pass
		and local hot spot filter works best.\\
	
	Blur image (Gaussian, ideal, Butterworth)			& $\boxed{\checkmark}$&
		Works well\\
	
	Sharpen image (Gaussian, ideal, Butterworth)	& $\boxed{\checkmark}$&
		Sub-optimal results, but is being sharpened.\\
	
	Display real, imaginary and magnitude of image.				& $\boxed{\checkmark}$&
		Grabbed during loading of image.\\
	
	Compare original and filtered image for similarities.	& $\boxed{\checkmark}$&
	Done with the compare function, but needs improvement.	\\[2mm]\hline
\end{tabular}
	\caption{Summary of tasks}\label{tab:summary}
\end{table}




\section{Future Work}\label{sec:futurework}
\subsection{Implement more metrics for comparison}
Current metrics is based around differences in the histograms generated by the
compare function and RMSE.  Implementing further methods of evaluating the
altered vs the original images would give a better understanding of how much
improvement the altercation has made the image. Mainly The filters would be
Blind Image Quality Assessments which don't rely upon a an image to compare it
to for generating the metrics.

A few direct suggestions for improved comparison metrics are:
\begin{itemize}
	\item Focus assessment.  By calculating the magnitude of the high-pass filter
		of an image. It gives you the metric which is utilised for measuring the
		focus.  The higher value, the better focus. and is calculated by:
		\begin{align}
			\Im_a &= \Im^{-1}(\Im(I) * H_{hp}) \\
			focus &= \Sigma | \Im_t( \Im( \Re(\Im_a) ) )|^2
		\end{align}

	\item Natural Scene Statistics.  Which generates a value for how the image
		compares against a natural scene using AI and blind image quality analysis.

	\item Contrast assessment.  By check subset of the area and calculating the
		contrast within that area. one can generate a method for comparing the
		images' based on the contrast relationship that exists in the image and also
		how much the altered has changed from the original image.

		Eg. the contrast of a subset can be computed by finding the difference
		between surrounding pixels and the area which is being checked. Eg. as shown
		below with regular contrast assessment and Weber ratio, where $a$ and $b$ is
		the median value of their respective areas.
		\begin{align}
			C_{regular} &= \frac{|a-b|}{a+b} 		\\
			Weber			&= \frac{|a-b|}{1+b} 			\\
			C_{weber} &= \frac{Weber}{0.75*Weber }
		\end{align}

	\item Feature assessment. By looking for similar constructs in the image and
		evaluating the shifts or disappearance one can assess parts of the image
		comparisons.

	\item Gray Scale Utilisation. This metric computes the entropy of the image
		and gives the value representing number of bits required to represent the
		image. This is used eg. to compute the level of entropy of the iris which
		is then used as basis for determining the image quality.
\end{itemize}



\subsection{Improve line filters}
	The line filter applied in this program is not exceptionally well made. The
	line filter only works on horizontal and vertical lines in the frequency
	spectrum.  Which means lines at an angle won't be detected.

\subsection{Improve line rotation filter}
	The line rotation filter works adequately, but are limited by not reaching the
full length of the matrix when it is rotated because the initial filter only
resides at an $0^\circ$ angle which is far shorter than the length required for
reaching the end of an $45^\circ$ corner of the matrix.


