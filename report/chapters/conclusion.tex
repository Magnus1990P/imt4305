\chapter{Conclusion}\label{sec:conclusion}



\section{Future Work}\label{sec:futurework}
This section metions some proposals for future work.

\subsection{Implement more metrics for comparison}
CUrrent metrics is based around differences in the histograms generated by the
compare function and RMSE.  Implementing further methods of evaluating the
altered vs the original images would give a better understanding of how much
improvement the altercation has made the image. Mainly THe filters would be
Blind Image Quality Asessments which don't rely upon a an image to compare it
to for genearting the metrics.

A few direct suggestions for improved comparison metrics are:
\begin{itemize}
	\item Focus assessment.  By calculating the magnitude of the high-pass filter
		of an image. It gives you the metric which is utilised for measuring the
		focus.  The higher value, the better focus. and is calculated by:
		\begin{align}
			\Im_a &= \Im^{-1}(\Im(I) * H_{hp}) \\
			focus &= \Sigma | \Im_t( \Im( \Re(\Im_a) ) )|^2
		\end{align}

	\item Natural Scene Statistics.  Which generates a value for how the image
		compares against a natural scene using AI and blind image quality analysis.

	\item Contranst assessment.  By check subset of the area and calculaitng the
		contrast within that area. one can generate a method for comparing the
		images' nased on the contrast relationship that exists in the image and also
		how much the altered has changed from the original image.

		Eg. the contrast of a subset can be computed by finding the difference
		between surrounding pixels and the area which is being checked. Eg. as shown
		below with regular contrast assessment and weber ratio, where $a$ and $b$ is
		the median value of their respective areas.
		\begin{align}
			C_{regular} &= \frac{|a-b|}{a+b} 		\\
			Weber			&= \frac{|a-b|}{1+b} 			\\
			C_{weber} &= \frac{Weber}{0.75*Weber }
		\end{align}

	\item Feature assessment. By looking for similar constructs in the image and
		evaluating the shifts or disapperance one can assess parts of the image
		comparisons.
\end{itemize}



\subsection{Improve line filters}
	The line filter applied in this program is not exeptionally well made. The
	line filter only works on horisontal and vertical lines in the frequency
	spectrum.  Which means lines at an angle won't be detected.

\subsection{Improve line rotation filter}
	The line rotation filter works adequately, but are limited by not reaching the
full length of the matrix when it is rotated because the initial filter only
resides at an $0^\circ$ angle which is far shorter than the length required for
reaching the end of an $45^\circ$ conrner of the matrix.


